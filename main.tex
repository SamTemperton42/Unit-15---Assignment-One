\documentclass[twocolumn, twoside, titlepage]{article}
\usepackage[utf8]{inputenc}
\usepackage{fontenc}
\title{Unit 15 - Assignment One}
\author{S0714607 }
\date{October 2018}

\begin{document}

\maketitle

\section{The Scenario}
You are working for a software development company as a computer programmer and have been given the task of developing a new computerised banking system for the client Kaleez bank, which operates in the Arabian Gulf using object oriented tools and technologies.  
\begin{itemize}
\item The Khajalee bank is a new bank authorised by the Cooperative Council for Arabian states to handle a single new currency, the Khajalee (K), as well as foreign exchanges for customers. 
\item In each country where Khajalee operates, there is a main head office and several branches in any one region. 
\item Each region has its own computer system, and each branch has its own computer system.
\item A branch reports to one regional head office.
\item A customer can have an account in any one of Khajalee’s branches. Accounts held by a customer are of two types; current and savings. Account details include an account number, a branch code, currency and balance.  
\item Customer details include the customer ID, name, address, postcode and contact number. 
\item A current account has an approved overdraft limit, and the date the account was approved.
\item A savings account can be one of three types; gold, silver or standard. 
\item The type of the account is determined by the opening amount of money deposited.
\item Finally, the interest rates for each type of account are Gold – 10 percent, silver – 7 percent, and standard – 3 percent. The interest is credited to each savings account at the beginning of each quarter.
\end{itemize}
Your manager needs to hire 2 junior programmer and prepared a set of questions to be used during the interview day and you have been given the responsibility to develop answer templates for those questions. 
\\You need to ensure, the answers appropriately cover the theory and practical aspect of the concepts asked for. 
\\You have been asked to design a demo application project template including a design template that will enable employees at the Khajalee bank to create new accounts for customers. \\ The application should also allow employees to deposit money on behalf of customers, check the current balance of a customer’s account, and transfer money between accounts, edit/delete accounts.
\clearpage

\section{Pass 1, Merit 1}
Define what according to you is an object and a class in an object-oriented program? How do you differentiate between the two? Give real life examples to explain the difference between both.

\subsection{Objects}
Object oriented programming, or OOP, revolves around objects. Much like in real life, everything is an object, and all that differentiates each object from one another is their set properties. These properties may include (continuing the real life analogy) whether the object is natural or man-made, sentient or insentient, species, sex, gender; the list goes on. Each object can have a near infinite amount of properties set to it. 

\subsection{Class}
In object oriented programming, a class is one of the most important concepts. This is because without classes, there would be no objects. For example, you want to classify a dog as an object. To do this, you first would use the class "dog". In this class, you would define a basic dog; breed, age, sex and fur colour (for example). Then, when you are creating a dog, you assign values to each of those properties, for example; a Labrador, that is 6 years old, male and has black fur. These values only exist in this one specific dog. So, you could then create a second dog, setting different values. This concept is also known as encapsulation (explained in more detail in section [3.5])

\subsection{Differentiating between Objects and Classes}


 %Give example screen-print codes of the below techniques . Label/annotate  each example, identifying where exactly you can see the the above tools and techniques are used.   

\subsection{Reusable units of Programming Logic}

\subsection{Data Abstraction}

\subsection{Inheritance}

\subsection{Polymorphism}

\subsection{Encapsulation}

\subsection{Importance}
In this section I shall be explaining the importance of each of these concepts in relation to object oriented programming as a whole.
\subsubsection{Reusable units of Programming Logic}

\subsubsection{Data Abstraction}

\subsubsection{Inheritance}

\subsubsection{Polymorphism}

\subsubsection{Encapsulation}


\section{Pass 2, Merit 2}
How do you demonstrate the following object oriented tools and techniques in your program? Explain why the tools and techniques would be used in production of ‘Khaleez Bank’ application.
\subsection{Declaring class using visibility}

\subsection{Creating new objects}

\subsection{Getter and Setter methods}

\subsection{Constructors}

\subsection{Class Libraries}

\subsection{Method Signature}

\subsection{Main Method}

\subsection{Overloading Methods}


\section{Pass 3}
Using an object oriented approach, produce a detailed design that meets the user’s requirements for the Khaleez bank. 
\subsection{Use Case Diagram}

\subsection{Class Diagram}

\subsection{User Interface Design (Including GUI and CLI)}


\end{document}
